%README file for moduleDocumentationTemplate TeX template.
%This template should be used to document all Basilisk modules.
%Updated 20170711 - S. Carnahan
%
%-Copy the contents of this folder to your own _Documentation folder
%
%-Rename the Basilisk-moduleDocumentationTemplate.tex appropriately
%
%-All edits should be made in one of:
%—header.tex
%— modelAssumptionsLimitations.tex
%— modelDescription.tex
%— modelFunctions.tex
%— revisionTable.tex
%— testDescription.tex
%— testParameters.tex
%— testResults.tex
%— user_guide.tex
%
%-Some rules about referencing within the document:
%1. If writing the suer guide, assume the module description is present
%2. If writing the validation section, assume the module features section is present
%3. Make no other assumptions about any sections being present. This allow for sections of the document to be used elsewhere without breaking.

%In order to import some of these sections into a document in a different directory:
%\usepackage{import}
%Then, the sections are called with \subimport{relative path}{file} in order to \input{file} using the right relative path.
%\import{full path}{file} can also be used if absolute paths are preferred over relative paths.

%%%%%%%%%%%%%%%%%%%%%%%%%%%%%%%%%%%%%%%%%%%%%%%%%%%%%%%%%%%%%%%%%%%%%%%%%%%%%
%%%%%%%%%%%%%%%%%%%%%%%%%%%%%%%%%%%%%%%%%%%%%%%%%%%%%%%%%%%%%%%%%%%%%%%%%%%%%
%%%%%%%%%%%%%%%%%%%%%%%%%%%%%%%%%%%%%%%%%%%%%%%%%%%%%%%%%%%%%%%%%%%%%%%%%%%%%


\documentclass[]{BasiliskReportMemo}

\usepackage{cite}
\usepackage{AVS}
\usepackage{float} %use [H] to keep tables where you put them
\usepackage{array} %easy control of text in tables
\usepackage{import} %allows for importing from multiple sub-directories
\bibliographystyle{plain}


\newcommand{\submiterInstitute}{Autonomous Vehicle Simulation (AVS) Laboratory,\\ University of Colorado}


\newcommand{\ModuleName}{GravityEffector}
\newcommand{\subject}{Gravity Effector C++ model}
\newcommand{\status}{Tested}
\newcommand{\preparer}{J.C. Sanchez}
\newcommand{\summary}{The gravity effector module is responsible for calculating the effects of gravity from a body on a spacecraft. Both spherical harmonics and polyhedron gravity implementations are developed and described. A unit test has been written and run which tests basic input/output, single-body gravitational acceleration, and multi-body gravitational acceleration.}

\begin{document}

\makeCover

%
%	enter the revision documentation here
%	to add more lines, copy the table entry and the \hline, and paste after the current entry.
%
\pagestyle{empty}
\clearpage
{\renewcommand{\arraystretch}{2}
	\noindent
	\begin{longtable}{|p{0.5in}|p{3.5in}|p{1.07in}|p{0.9in}|}
		\hline
		{\bfseries Rev} & {\bfseries Change Description} & {\bfseries By}& {\bfseries Date} \\
		\hline
		1.0 & First version - Mathematical formulation and implementation & M. Diaz Ramos & 2017-01-01\\
		\hline
		1.1 & Added test documentation & S. Carnahan &2017-07-13\\
		\hline
		1.2 & Added documentation on the multi-body gravity acceleration calculation & S. Carnahan &2018-07-20\\
		\hline
		1.3 & Added documentation on the polyhedron model gravity acceleration computation & J.C. Sanchez &2022-07-05\\
		\hline
		1.4 & Updated information for latest refactor and removed outdated information & J. Garcia Bonilla &2023-05-17\\
		\hline
	\end{longtable}
}

%\newpage
\setcounter{page}{1}
\pagestyle{fancy}

\tableofcontents %Autogenerate the table of contents
~\\ \hrule ~\\ %Makes the line under table of contents
	
\section{Model Description}
There two types of RW control torque interfaces, analog and digital. This modules assumes the RW is controlled through a set of voltages sent to the RW motors.  This module is developed in a general manner where a voltage deadband is assumed and the module can be run in a pure open-loop manner, or with a closed-loop torque tracking control mode.  Finally, if a RW availability message is present, then the RW is set to zero if the corresponding availability is set to {\tt UNAVAILABLE}. 



\begin{figure}[htb]
	\centerline{
		\includegraphics[]{Figures/us2V}
	}
	\caption{Illustration of RW motor torque to voltage conversion}
	\label{fig:us2V}
\end{figure}
\subsection{Open-loop voltage conversion}
This module requires the RW configuration message to contain the maximum RW motor torque values $u_{\text{max}}$.  The user must specify the minimum and maximum output voltages as shown in Figure~\ref{fig:us2V}.  The minimum voltage is a voltage below which the motor doesn't apply a torque, i.e. a deadzone.  

Let the intermediate voltage value $V_{\text{int}}$ as
\begin{equation}
\label{eq:rwMV:1}
V_{\text{int}} = \frac{V_{\text{max}} - V_{\text{min}}}{u_{\text{max}}} u_{s}
\end{equation}
The output voltage is thus determined through
\begin{equation}
V = V_{\text{int}} + V_{\text{min}} *\text{sgn}(V_{\text{int}} )
\end{equation}

\subsection{RW Availability} 
If the input message name {\tt rwAvailInMsg} is defined, then the RW availability message is read in. The voltage mapping is only performed if the individual RW availability setting is {\tt AVAILABLE}.  If it is {\tt UNAVAILABLE} then the output voltage is set to zero.


\subsection{Closed-loop commanded torque tracking}
The requested RW motor torque is given by $u_{s}$.  The RW wheel speed $\Omega$ is monitored to see if the actual torque being applied matches the commanded torque.  Let $J_{s}$ be the RW spin inertia about the RW spin axis $\hat{\bm g}_{s}$.   In the following development the motor torque equation is approximated as
\begin{equation}
u_{s} = J_{s} \dot\Omega
\end{equation}
where the assumption is made that the spacecraft angular accelerations are small compared to the RW angular accelerations.  The $\dot\Omega$ term is digitally evaluated using a backwards difference method:
\begin{equation}
\dot\Omega_{n} = \frac{\Omega_{n} - \Omega_{n-1}}{\Delta t}
\end{equation}
Care is taken that the old RW speed information $\Omega_{n-1}$ is not used unless a history of wheel speeds is available, in particular, after a module reset.  Thus, the actual RW torque is evaluated as
\begin{equation}
u_{n} = J_{s} \dot\Omega_{n}
\end{equation}
Finally, the closed loop motor torque value is computed with a proportional feedback component as
\begin{equation}
u_{s,CL} = u_{s} - K (u_{n} - u_{s})
\end{equation}
where $K>0$ is a positive feedback gain value.  Finally, this $u_{s,CL}$ is fed to the voltage conversion process in Eq.~\eqref{eq:rwMV:1}.  

\subsection{Saturation and Dead Band}
If the calculated voltage is outside of $\pm V_{\mathrm{max}}$, then the voltage is saturated at the $\pm V_{\mathrm{max}}$ value. Note, this corresponds to the reaction wheel torques being saturated. Similarly, if the calculated voltage is inside $\pm V_{\mathrm{min}}$, then the voltage is set to $\pm V_{\mathrm{min}}$. This simulates the dead band. If the $V_{\mathrm{min}} = 0$, then there is no dead band. %This section includes mathematical models, code description, etc.

\section{Model Functions}
The mathematical description of gravity effects are implemented in gravityEffector.cpp. This code performs the following primary functions
\begin{itemize}
	\item \textbf{GravBody Creation}: The code creates gravity bodies which are capable of affecting spacecraft. It does not effect a spacecraft unless that spacecraft explicitly adds the body as a gravity effector.
	\item \textbf{Orbital Energy}: The code can calculate the total orbital energy as well as orbital kinetic and orbital potential energy of a spacecraft about a gravity body.
	\item \textbf{Compute gravity}: The code can use different gravity models to compute the gravity of a body. The following models are implemented in Basilisk:
	\begin{itemize}
		\item \textbf{Simple Gravity}: The code can compute a gravity acceleration between two bodies according to Newton's law of universal gravitation given $\mu$ and the distance between the bodies.
		\item \textbf{Spherical Harmonics}: The code can compute gravity acceleration between two bodies using the more-complex method of spherical harmonics. To do this, it must be provided with the same inputs as for calculating simple gravity. In addition, it needs to be provided a "degree" of spherical harmonics to be used and spherical harmonics coefficients useful up to that degree.
		\item \textbf{Polyhedral:} The code can compute gravity acceleration between two bodies using the polyhedral model. To do this, it must be provided with the same inputs as for calculating simple gravity. In addition, it needs to be provided with the vertexes positions and their assignment to faces. 
	\end{itemize}
	\item \textbf{Multiple Body Effects}: The code can stack the effects of multiple gravity bodies on top of each other to determine the net effect on a spacecraft. The user must indicate in the spacecraft set-up which gravitational bodies should be taken into account.
	\item \textbf{Interface: Spacecraft States}: The code sends and receives spacecraft state information via the DynParamManager.
	\item \textbf{Interface: Energy Contributions}: The code sends spacecraft energy contributions via \\updateEnergyContributions() which is called by the spacecraft.
	\item \textbf{Interface: GravBody States}: The code outputs GravBody states(ephemeris information) via the Basilisk messaging system.
	
\end{itemize}

\section{Model Assumptions and Limitations}
\subsection{Spherical harmonics gravity model}
The limitations of spherical harmonics gravity model are well-known and clearly explained in Schaub and Junkins' book\cite{schaub2014}. The limitations include:
\begin{itemize}
	\item \textbf{Coefficient Accuracy}: The coefficients used in the spherical harmonics equations are typically calculated based on gravitational data gathered by satellites. Therefore, the accuracy of the model is determined by the accuracy of the satellite instrumentation and precision of the stored data. Furthermore, for some bodies, there may not be sufficient information available to provide accurate coefficients or higher-degree coefficients.
	\item \textbf{Maximum Degree}: The spherical harmonics equation is a series expansion. Therefore, any implementation must truncate the equation at some point. The truncated portion of the equation necessarily defines some amount of error in the final calculation. This error is, however, small after the first handful of terms. Additionally, a larger distance between gravity body and spacecraft requires fewer terms of the series to achieve equal accuracy as compared to a case with less distance. This code allows the user to request a maximum number of terms to evaluate rather than a specific accuracy. This could lead to less-than-desirable accuracy with small separation distances and greater-than-necessary run times with large separation distances.
	\item \textbf{Planetary Ephemeris Data}: This code generally relies on an external package for planetary ephemeris information. Errors included in this package will translate into error in the gravity calculations, but those errors should be small. Because the ephemeris data is tabulated, this code should not be used to try to project the orbits of the celestial bodies in question. This could be done, though, by treating any celestial body as a "spacecraft".
\end{itemize}

\subsection{Polyhedral gravity model}
The limitations of polyhedral gravity model are well-known and clearly explained in Werner and Scheere's article\cite{werner1996}. The limitations include:
\begin{itemize}
	\item \textbf{Constant Density}: The polyhedron gravity computation assumes that the body has constant density. Consequently, this method does not account for spatial density variations that typically arise within the internal structure of bodies or in contact binary asteroids.
	\item \textbf{Shape Accuracy}: The polyhedral model assumes the body shape is described as a polyhedron which is an approximation of the continuous real shape. The resolution of the model can be augmented by increasing the number of vertexes and faces though, in turn, this may considerably slow down the gravity evaluation times. Let recall that the polyhedron gravity computation requires to loop over all faces and edges.
	\item \textbf{Trisurface Polyhedron:} The implemented computation is case-specific for polyhedrons with faces composed of three vertexes. This reduces the possible polyhedrons to a single geometrical topology. However, the trisurface polyhedron is the standardized shape for small bodies.     
\end{itemize}


 %This includes a concise list of what the module does. It also includes model assumptions and limitations

% !TEX root = ./Basilisk-atmosphere-20190221.tex

\section{Test Description and Success Criteria}
This section describes the specific unit tests conducted on this module.

\subsection{General Functionality}

\subsubsection{setDensityMessage}

This test verifies that the user can specify the atmospheric density message used by the module.

\subsubsection{testDragForce}

This test verifies that the module correctly calculates the drag force given the model's assumptions. It also implicitly tests the compatibility of facetDrag and exponentialAtmosphere. 

\subsubsection{testShadow}

This test verifies that panels that are not in the flow are correctly ignored for the purposes of drag calculation.

\subsection{Model-Specific Tests}

\subsubsection{test\_unitFacetDrag.py}

This unit test runs setDensityMessage, testDragForce, and testShadow to verify the functionality of the module. 

\section{Test Parameters}
The simulation tolerances are shown in Table~\ref{tab:errortol}.  In each simulation the neutral density output message is checked relative to python computed true values.  
\begin{table}[htbp]
	\caption{Error tolerance for each test.}
	\label{tab:errortol}
	\centering \fontsize{10}{10}\selectfont
	\begin{tabular}{ c | c } % Column formatting, 
		\hline\hline
		\textbf{Output Value Tested}  & \textbf{Tolerated Error}  \\ 
		\hline
		{\tt newDrag.forceExternal\_N}        & \input{AutoTeX/toleranceValue} (relative)   \\ 		\hline\hline
	\end{tabular}
\end{table}




\section{Test Results}
The following table shows the test result.


\begin{table}[H]
	\caption{Test result for test\_unitFacetDrag.py}
	\label{tab:results}
	\centering \fontsize{10}{10}\selectfont
	\begin{tabular}{c  | c } % Column formatting, 
		\hline\hline
		\textbf{Check} &  \textbf{Pass/Fail} \\ 
		\hline
		1 &  \input{AutoTeX/unitTestPassFail} \\ 
		\hline
		\hline
	\end{tabular}
\end{table}



 % This includes test description, test parameters, and test results

% !TEX root = ./Basilisk-ThrusterForces-20160627.tex

\section{User Guide}
\begin{enumerate}


\item \textbf{$\epsilon$ Parameter}: The minimum norm inverse requires a non-zero determinant value of $[D][D]^{T}$.  For this setup, this matrix is a scalar value
\begin{equation}
	D_{2} = \text{det}([D][D]^{T})
\end{equation}
If this $D_{2}$ value is near zero, then the full 3D $\bar{\bm L}_{r}$ vector cannot be achieved.  A common example of such a scenario is with the DV thruster configuration where all $\hat{\bm g}_{t_{i}}$ axes are collinear.  Torques about these thrust axes cannot be produced.  In this case, the minimum norm solution is adjusted to only match the torques along the control matrix $[C]$ sub-space.  Torques being applied outside of $\hat{\bm c}_{j}$ is not possible as $D_{2}$ is essentially zero, indicating the other control axis cannot be controlled with this thruster configuration.  

The minimum norm torque solution is now modified to use
\begin{equation}
	\bar{\bm F} = ([C][\bar D])^{T}( [C][\bar D][\bar D]^{T}[C]^{T})^{-1} [C] {\bm L}_{r}
\end{equation}
As the thruster configuration cannot produce a general 3D torque, here the $[C]$ matrix must have either 1 or 2 control axes that are achievable with the given thruster configuration.  

To set this epsilon parameter, not the definition of the $[D]$ matrix components $\bm d_{i} = (\bm r_{i} \times \hat{\bm g}_{t_{i}})$. Note that $\bm r_{i} \times \hat{\bm g}_{t_{i}}$ is a scaled axis along which the $i^{\text{th}}$ thruster can produce a torque.  The value $\bm d_{i}$ will be near zero if the dot product of this axis with the current control axis $\hat{\bm c}_{j}$ is small.  

To determine an appropriate $\epsilon$ value, let $\alpha$ be the minimum desired angle to avoid the control axis $\hat{\bm c}_{j}$ and the scaled thruster torque axis $\bm r_{i} \times \hat{\bm g}_{t_{i}}$ being orthogonal.  If $\bar r$ is a mean distance of the thrusters to the spacecraft center of mass, then the $d_{i}$ values must satisfy
\begin{equation}
	\frac{d_{i}}{\bar r} > \cos(90\dg - \alpha) = \sin\alpha
\end{equation}
Thus, to estimate a good value of $\epsilon$, the following formula can be used
\begin{equation}
	\epsilon \approx d_{i}^{2} = \sin^{2}\!\alpha \ \bar{r}^{2}
\end{equation}
For example, if $\bar{r} = 1.3$ meters, and we want $\alpha$ to be at least 1$\dg$, then we would set $\epsilon = 0.000515$.

\item \textbf{$[C]$ matrix}: The module requires control control axis matrix $[C]$ to be defined.  Up to 3 orthogonal control axes can be selected.  Let $N_{c}$ be the number of control axes.  The $N_{c}\times 3$ $[C]$ matrix is then defined as
\begin{equation}
	[C] = \begin{bmatrix}
		\hat{\bm c}_{1}
		\\
		\vdots
	\end{bmatrix}
\end{equation}

Not that in python the matrix is given in a 1D form by defining {\tt controlAxes\_B}.  Thus, the $\hat{\bm c}_{j}$ axes are concatenated to produce the input matrix $[C]$. 

\item \textbf{\tt thrForceSign Parameter}: Before this module can be run, the parameter {\tt thrForceSign} must be set to either +1 (on-pulsing with the ACS configuration) or -1 (off-pulsing with the DV configuration).

\item \textbf{\tt use2ndLoop} Flag: If the  {\tt thrForceSign} flag is set to +1 then an on-pulsing configuration is setup.  By default the optional flag  {\tt use2ndLoop} is 0 and the algorithm  only uses the least-squares fitting loop once.  By setting the {\tt use2ndLoop} to +1 then the 2nd least squares fitting loop is used during this on-pulsing configuration,. 

\item \textbf{{\tt angErrThresh} Parameter}: The default value of {\tt angErrThresh} is 0\dg.  This means that during periods of thruster saturation the thruster force solution $\bm F$ is scaled such that $|F_{i}| \le F_{\text{max}}$.  If this scaling should only be done if $\bm\tau$ and $\bar{\bm L}_{r}$ differ by an angle $\alpha$, then set {\tt angErrThresh} equal to $\alpha$.  To turn off this force scaling during thruster saturation the parameter   {\tt angErrThresh} should be set to a value larger than 180\dg.  

\end{enumerate} % Contains a discussion of how to setup and configure  the BSK module






\bibliography{bibliography} %This includes references used and mentioned.

\end{document}
